\documentclass[utf8]{beamer}
\mode<presentation>
\usepackage{listings}
\usepackage{helvet}
\usetheme{Warsaw}
\usecolortheme{whale}
\usefonttheme[onlylarge]{structuresmallcapsserif}
\usefonttheme[onlysmall]{structurebold}

\setbeamercovered{dynamic}
\setbeameroption{show notes}

\begin{document}
\title{Scaling Erlang Web Applications}
\subtitle{100 to 100K users at one web server}
\author{Fernando Benavides (\textit{@elbrujohalcon})}
\institute{Inaka Labs}
\date{\today}

\frame{\titlepage} 

\begin{frame}
	\frametitle{Hello World!}
	\begin{itemize}
		\item<+-> I'm a developer since I was 10
		\item<+-> I'm an Erlang developer since 2008
		\item<+-> I've worked in many dynamic web sites
		\item<+-> Most of them with high scale requirements
		\item<+-> I'll share my experience with you 
	\end{itemize}
\end{frame}

\frame{\frametitle{Outline}\tableofcontents[pausesections]} 

\section{The Challenge}
\subsection{Description}
\begin{frame}
	We will work on the scalability of a \emph{web} project \pause that has a \emph{HTTP API} \pause and keeps clients \emph{connected} to the server \pause for \emph{long periods} of time.\\
\pause Examples:
	\begin{itemize}
		\item Social sites
		\item Chat sites
		\item Sports sites
	\end{itemize}
\end{frame}

\subsection{Scope}
\begin{frame}
	\emph{We will focus on}
	\begin{itemize}
		\item OTP behaviours
		\item TCP connections
		\item mochiweb
		\item Underlaying system configurations
	\end{itemize}
	\pause
	\emph{We will \textbf{not} deal with}
	\begin{itemize}
		\item Multiple machines/nodes
		\item Databases
	\end{itemize}
\end{frame}

\section{The Plan}
\begin{frame}
	\center{\textsc{\huge{The Plan}}}
\end{frame}

\begin{frame}{General Considerations}
	\begin{itemize}
		\item<1,5> Create a system that \alert{works}
		\item<2,5> Automate your clients
		\item<3,5> Keep a human watching
		\item<4,5> Be patient
	\end{itemize}
\end{frame}

\subsection{Finding The Initial Boundaries}
\begin{frame}{Goals}
	\begin{itemize}
		\item Test the system as it is
		\item How many users can the system handle \alert{as is}?
		\item Find $N$ and $C$
	\end{itemize}
\end{frame}
\begin{frame}{Steps}
	\begin{itemize}
		\item Choose $N$ and $C$
		\item Test the API
		\item Test long-lived connections
		\item Test both
		\pause
		\item Repeat with higher values for $N$ and $C$
	\end{itemize}
\end{frame}

\subsection{Blackbox Tests}
\begin{frame}{Goals}
	\begin{itemize}
		\item Improve the environment
		\item Tune-In the machine(s)
		\item \alert{Don't} touch the code
	\end{itemize}
\end{frame}
\begin{frame}{Steps}
	\begin{itemize}
		\item Check kernel variables
		\item Check system limits
		\item Check Erlang VM parameters
	\end{itemize}
\end{frame}

\subsection{Erlang Tuning}
\begin{frame}{Goals}
	\begin{itemize}
		\item Tune up \alert{your} system
		\item Discover scalability issues and fix them
		\item Find the biggest $N$ and $C$ for \alert{one node}
	\end{itemize}
\end{frame}
\begin{frame}{Steps}
	\begin{itemize}
		\item Choose $N$ and $C$ to fail
		\item Find the problem
		\item Fix it
		\item Add it to the list of \emph{Tips and Tricks}
		\pause
		\item Repeat with higher values for $N$ and $C$
	\end{itemize}
\end{frame}

\subsection{Adding Nodes}
\begin{frame}{Goals}
	\begin{itemize}
		\item Get the system ready to work on many nodes
		\item Decide if they should be connected or not
		\item Find the $N$ and $C$ \alert{per node}
	\end{itemize}
\end{frame}
\begin{frame}{Steps}
	\begin{itemize}
		\item Get the second node running
		\item Choose $N$ and $C$
		\item Try interconnected instances
		\item Try independent instances
		\pause
		\item Repeat with higher values for $N$ and $C$
	\end{itemize}
\end{frame}

\section{Tips and Tricks}
\subsection{TCP Tunning}
\begin{frame}{OS tweaks}
	TODO: Copy from the article
\end{frame}
\begin{frame}{Erlang tweaks}
	TODO: Copy from the article on listeners
	TODO: Copy from the article on inbound TCP connections
	TODO: Copy from the article on outbound TCP connections
\end{frame}
\subsection{OTP}
\begin{frame}{gen\textunderscore event}
	TODO: Copy from the article on sup\textunderscore handler
	TODO: Copy from the article on long delivery queues
\end{frame}
\begin{frame}{gen\textunderscore servers}
	TODO: Copy from the article on timing out
	TODO: Copy from the article on too much memory
	TODO: Copy from the article on taking too long to initiliaze
\end{frame}
\begin{frame}{supervisors}
	TODO: Copy from the article
\end{frame}
\subsection{Other Stuff}
\begin{frame}{Process Registration}
	TODO: Copy from the article
\end{frame}
\begin{frame}{Timers}
	TODO: Copy from the article
\end{frame}
\begin{frame}{Logging}
	TODO: Copy from the article
\end{frame}

\section{Final Words}
\subsection{Summary}
\begin{frame}{Summary}
	TODO: Summary
\end{frame}
\subsection{Other stuff}
\begin{frame}{Other stuff}{that we left out of this presentation}
	TODO: List of other scalability stuff we left out
\end{frame}
\subsection{Questions}
\begin{frame}
	\alert{Any questions?}
\end{frame}

\appendix

\lstset{language=erlang}
\defverbatim[colored]\mycode{%
\begin{lstlisting}[frame=single, emph={fact}, emphstyle={\color{blue}}]
-spec fact(integer()) -> integer().
fact(N) ->
	lists:fold(fun(X, F) ->
			F * X
		   end, 1, lists:seq(1,N)).
\end{lstlisting}
}

\section{Some Code}
\begin{frame}
\mycode
\end{frame}

\end{document}