\documentclass[utf8]{beamer}
\mode<presentation>
\usepackage{listings}
\usepackage{helvet}
\usetheme{Warsaw}
\usecolortheme{whale}
\usefonttheme[onlylarge]{structuresmallcapsserif}
\usefonttheme[onlysmall]{structurebold}

\setbeameroption{show notes}

\begin{document}
\title{Scaling Erlang Web Applications}
\subtitle{100 to 100K users at one web server}
\author{Fernando Benavides (\textit{@elbrujohalcon})}
\institute{Inaka Labs}
\date{\today}

\frame{\titlepage} 

\begin{frame}
	\frametitle{Who am I?}
	TODO: some funny stuff about Argentina, me, Erlang, elbrujohalcon... maybe some pictures
	\note{Brief review of my story, why am I an Erlang programmer, how much I know about web applications and scalability}
\end{frame}

\begin{frame}
	\frametitle{About Inaka}
	TODO: Inaka's Info
	\note{Brief review of Inaka's story, the systems we develop and why scalability matters to us}
\end{frame}

\frame{\frametitle{Outline}\tableofcontents[pausesections]} 

\section{The Challenge}
\subsection{Description}
\begin{frame}
	We will work on the scalability of a \emph{web} project \pause that has a \emph{HTTP API} \pause and keeps clients \emph{connected} to the server \pause for \emph{long periods} of time.
\end{frame}
\begin{frame}
	We will work on the scalability of a \emph{web} project that has a \emph{HTTP API} and keeps clients \emph{connected} to the server for \emph{long periods} of time.
	For example:
	\begin{itemize}
		\item Social sites
		\item Chat sites
		\item Sports sites
	\end{itemize}
\end{frame}

\subsection{Scope}
\begin{frame}
	\emph{We will deal with}
	\begin{itemize}
		\item OTP behaviours
		\item TCP connections
		\item mochiweb
		\item Underlaying system configurations
	\end{itemize}
	\pause
	\emph{We will \textbf{not} deal with}
	\begin{itemize}
		\item Multiple machines/nodes
		\item Databases
	\end{itemize}
\end{frame}

\section{The Plan}
\subsection{Is it really working?}
\begin{frame}{Goals}
	TODO: this stage goals
\end{frame}
\begin{frame}{Steps}
	TODO: this stage steps
\end{frame}

\subsection{Finding The Boundaries}
\begin{frame}{Goals}
	TODO: this stage goals
\end{frame}
\begin{frame}{Steps}
	TODO: this stage steps
\end{frame}

\subsection{Blackbox Tests}
\begin{frame}{Goals}
	TODO: this stage goals
\end{frame}
\begin{frame}{Steps}
	TODO: this stage steps
\end{frame}

\subsection{Erlang Tuning}
\begin{frame}{Goals}
	TODO: this stage goals
\end{frame}
\begin{frame}{Steps}
	TODO: this stage steps
\end{frame}

\subsection{Adding Nodes}
\begin{frame}{Goals}
	TODO: this stage goals
\end{frame}
\begin{frame}{Steps}
	TODO: this stage steps
\end{frame}

\section{Tips and Tricks}
\subsection{TCP Tunning}
\begin{frame}{OS tweaks}
	TODO: Copy from the article
\end{frame}
\begin{frame}{Erlang tweaks}
	TODO: Copy from the article on listeners
	TODO: Copy from the article on inbound TCP connections
	TODO: Copy from the article on outbound TCP connections
\end{frame}
\subsection{OTP}
\begin{frame}{gen\textunderscore event}
	TODO: Copy from the article on sup\textunderscore handler
	TODO: Copy from the article on long delivery queues
\end{frame}
\begin{frame}{gen\textunderscore servers}
	TODO: Copy from the article on timing out
	TODO: Copy from the article on too much memory
	TODO: Copy from the article on taking too long to initiliaze
\end{frame}
\begin{frame}{supervisors}
	TODO: Copy from the article
\end{frame}
\subsection{Other Processes}
\begin{frame}{Process Registration}
	TODO: Copy from the article
\end{frame}
\begin{frame}{Timers}
	TODO: Copy from the article
\end{frame}
\begin{frame}{Logging}
	TODO: Copy from the article
\end{frame}

\section{Final Words}
\subsection{Summary}
\begin{frame}{Summary}
	TODO: Summary
\end{frame}
\subsection{Other stuff}
\begin{frame}{Other stuff}{that we left out of this presentation}
	TODO: List of other scalability stuff we left out
\end{frame}
\subsection{Questions}
\begin{frame}
	\alert{Any questions?}
\end{frame}

\appendix

\lstset{language=erlang}
\defverbatim[colored]\mycode{%
\begin{lstlisting}[frame=single, emph={fact}, emphstyle={\color{blue}}]
-spec fact(integer()) -> integer().
fact(N) ->
	lists:fold(fun(X, F) ->
			F * X
		   end, 1, lists:seq(1,N)).
\end{lstlisting}
}

\section{Some Code}
\begin{frame}
\mycode
\end{frame}

\end{document}