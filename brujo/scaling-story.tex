\documentclass[utf8]{beamer}
\mode<presentation>
\usepackage{listings}
\usepackage{helvet}
\usetheme{Warsaw}
\usecolortheme{whale}
\usefonttheme[onlylarge]{structuresmallcapsserif}
\usefonttheme[onlysmall]{structurebold}
\setbeamercovered{dynamic}

\begin{document}
\title{Scaling Erlang Web Applications}
\subtitle{100 to 100K users at one web server}
\author{Fernando Benavides (\textit{@elbrujohalcon})}
\institute{Inaka Labs}
\date{\today}
\logo{\includegraphics[height=0.5cm]{img/inaka_leaf_logo.png}}

\lstset{% general command to set parameter(s)
		basicstyle=\ttfamily\tiny, % print whole listing small
		keywordstyle=\color{black}\bfseries,
		identifierstyle=, % nothing happens
		stringstyle=\ttfamily, % typewriter type for strings
		showstringspaces=false} % no special string spaces

\frame{\titlepage}

\setbeamercolor{stamp}{fg=black,bg=yellow}

\begin{frame}
\only<1,2>{
	\begin{center}
		\textsc{{\Huge Inaka Networks}}
	\end{center}
\onslide<2>{
	\begin{center}
		presents \ldots
	\end{center}}}
\only<3,4>{
	\begin{center}
		\textit{{\LARGE El Brujo Halc\'on}}
	\end{center}
\onslide<4>{
	\begin{center}
		in \ldots
	\end{center}}}
\only<5,6>{
	\begin{center}
		\textsc{{\huge Scaling Erlang}}
	\end{center}
\onslide<6>{
	\begin{beamercolorbox}[right,wd=1\textwidth,shadow=true,rounded=true]{stamp}
		\emph{Based on a true story}
	\end{beamercolorbox}
	}}
\end{frame}

\begin{frame}
	\begin{center}
		{\Large
			\textsf{
				A \emph{not so} long time ago \pause
				\includegraphics<2>[width=.75\textwidth]{img/map.jpg}\\
				\only<2>{in a country far far away\ldots}}}
	\end{center}
\end{frame}

\begin{frame}
	\begin{description}
		\item<+->[A Friend]\ \\ Hey! Let's watch the \emph{supercl\'asico}!!!
		\item<+->[Brujo]\ \\ I can't, I'm at \emph{the office}
		\item<+->[A Friend]\ \\ \ldots
		\item<+->[Brujo]\ \\ We need an app for that!
	\end{description}
\end{frame}

\begin{frame}
	\begin{description}
		\item<+->[Brujo]\ \\ Let's call it \textsc{MatchStream}
		\item<+->[A Friend]\ \\ Ok, then\ldots We know there will be hundreds of thousands of users, right?\\ We need the system to \textbf{scale}
		\item<+->[Brujo]\ \\ Of course! We should use \alert{Erlang}!
	\end{description}
\end{frame}

\begin{frame}
	\begin{center}
		{\Large
			\textsf{A while later\ldots}}
	\end{center}
\end{frame}

\begin{frame}{MatchStream}{System Description}
\includegraphics[width=\textwidth]{img/MatchStream.png}
\end{frame}

\begin{frame}{MatchStream}{Architecture}
	\begin{center}
		\includegraphics[height=.75\textheight]{img/running.png}
	\end{center}
	TODO: Take this picture with client(s) connected
\end{frame}

\begin{frame}
	\begin{description}
		\item<+->[Brujo]\ \\ Boca plays again today, let's try our system out with this game!\\ What can \textbf{possibly} go wrong?
		\item<+->[User 1]\ \\ Wow! \textsc{MatchStream} is awesome!
		\item<+->[\ldots]\ 
		\item<+->[User 100]\ \\ Hey! this system is a total crap! It doesn't even let me connect to it!
		\item<+->[Brujo]\ \\ WTF?! \alert{The system doesn't scale}!!
		\item<+->[A Friend]\ \\ Didn't you use \alert{Erlang}?
	\end{description}
\end{frame}

\begin{frame}
	\begin{beamercolorbox}[wd=1\textwidth,shadow=true,rounded=true]{stamp}
		\emph{\textsc{Lesson Learned}}
	\end{beamercolorbox}
	\ \\
	\textbf{Just using Erlang is not enough to make your system scale}
\end{frame}

\begin{frame}
	\begin{center}
		{\Large
			\textsf{So, we made it scale\ldots}}
	\end{center}
\end{frame}

\begin{frame}{Step 1}
First of all we wanted to be sure that the system was actually working.\\
\pause
\begin{itemize}
	\item<+-> We built a simulator
	\item<+-> We improved the logging mechanisms
	\item<+-> We tested the system
	\item<+-> We found its initial scale limits
\end{itemize}
\end{frame}

\begin{frame}{Step 1}{Results}
	\includegraphics[top=-1,width=\textwidth]{img/MatchStream-1.png}
\end{frame}

\begin{frame}{Step 2}
Once we knew the system was fine, we decided to tune up the server where it was installed.
\pause
So, we checked the kernel variables and system limits for
\begin{itemize}
	\item<+-> Concurrent TCP connections
	\item<+-> Open files limit
	\item<+-> TCP backlog size
	\item<+-> TCP memory allocation
	\item<+-> Erlang VM process limit
\end{itemize}
\end{frame}

\begin{frame}{Step 2}{Results}
	\includegraphics[top=-1,width=\textwidth]{img/MatchStream-2.png}
\end{frame}

\begin{frame}[t]{Step 3}
Then we decided to start improving the different components of the system.\\
\pause
We called a friend to help us\ldots
	\begin{center}
		\includegraphics<3>[width=.65\textwidth]{img/macgyver.jpg}
	\end{center}
\end{frame}

\begin{frame}{Step 3}{Connection tweaks}
	\begin{description}
		\item<+->[Backlog]\ \\
			\begin{itemize}
				\item Allow more concurrent connections
				\item Remember HTTP \emph{runs on} TCP
			\end{itemize}
		\item<+->[Connections]\ \\
			\begin{itemize}
				\item Don't use just one of them
				\item Check inbound and outbound connections
			\end{itemize}
	\end{description}
\end{frame}

\begin{frame}
TODO users / TODO at a time
\end{frame}

\begin{frame}{Step 3}{gen\textunderscore event}
	\begin{description}
		\item<+->[sup\textunderscore handler]\ \\
			\begin{itemize}
				\item Don't use it
				\item Monitor the processes instead
			\end{itemize}
		\item<+->[Long Delivery Queues]\ \\
			\begin{itemize}
				\item Use \emph{repeaters}
			\end{itemize}
	\end{description}
\end{frame}

\begin{frame}
TODO users / TODO at a time
\end{frame}

\begin{frame}{Step 3}{gen\textunderscore server}
	\begin{description}
		\item<+->[Call Timeouts]\ \\
			Remember \texttt{gen\textunderscore server:reply/2}
		\item<+->[Memory Footprint]\ \\
			Remember \texttt{hibernate}
		\item<+->[Long \texttt{init/1}]\ \\
			Use $0$ timeout
	\end{description}
\end{frame}

\begin{frame}
TODO users / TODO at a time
\end{frame}

\lstset{language=erlang}
\defverbatim[colored]\startchild{%
\begin{lstlisting}[frame=single]
      supervisor:start_child(
      	list_to_atom("module-name_" ++
				integer_to_list(random:uniform(#ofSupervisors))).
\end{lstlisting}
}

\begin{frame}{Step 3}{supervisors}
	\begin{itemize}
		\item Sometimes \texttt{simple\textunderscore one\textunderscore for\textunderscore one} supervisors get \alert{overburdened} because they have too many children
		\item Try a supervisor hierarchy with several managers below the main supervisor
		\item Turn \texttt{supervisor:start\textunderscore child/2} calls into something like
		\startchild
	\end{itemize}
\end{frame}

\begin{frame}
TODO users / TODO at a time
\end{frame}

\begin{frame}{Step 3}{Other Processes}
	\begin{description}
		\item<+->[Timers]\ \\
			\begin{itemize}
				\item Don't use the \texttt{timer} module
				\item Use \texttt{erlang:send\textunderscore after}
			\end{itemize}
		\item<+->[Logging]\ \\
			\begin{itemize}
				\item Don't log too much
				\item Use a good logging system
			\end{itemize}
		\item<+->[Registration]\ \\
			\begin{itemize}
				\item Sometimes it's better to register processes instead of keeping track of their pids manually
				\item You can always register processes \alert{both} locally and globally
			\end{itemize}
	\end{description}
\end{frame}

\begin{frame}{Step 3}{Results}
	\includegraphics[top=-1,width=\textwidth]{img/MatchStream-3.png}
\end{frame}

\begin{frame}
TODO: Img of what the system looks like at this point
\end{frame}

\begin{frame}{Step 4}
Well, let's add some nodes to it!
\end{frame}

\begin{frame}{Step 4}{Adding Nodes}
Again, it's not as easy as just starting the app in another Erlang node
\pause
We needed to find the best topology, we considered using:
	\begin{itemize}
		\item connected nodes
		\item independent nodes
	\end{itemize}
\pause
We had to decide which processes needed to communicate and how
\pause
and of course, test the whole system again
\end{frame}

\begin{frame}{Step 4}{Results}
	\includegraphics[top=-1,width=\textwidth]{img/MatchStream-4.png}
\end{frame}

\end{document}