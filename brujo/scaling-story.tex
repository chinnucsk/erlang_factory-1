\documentclass[utf8]{beamer}
\mode<presentation>
\usepackage{listings}
\usepackage{helvet}
\usetheme{Warsaw}
\usecolortheme{whale}
\usefonttheme[onlylarge]{structuresmallcapsserif}
\usefonttheme[onlysmall]{structurebold}
\setbeamercovered{dynamic}

\begin{document}
\title{Scaling Erlang Web Applications}
\subtitle{100 to 100K users at one web server}
\author{Fernando Benavides (\textit{@elbrujohalcon})}
\institute{Inaka Labs}
\date{\today}
\logo{\includegraphics[height=0.5cm]{img/inaka_leaf_logo.png}}

\lstset{% general command to set parameter(s)
		basicstyle=\ttfamily\tiny, % print whole listing small
		keywordstyle=\color{black}\bfseries,
		identifierstyle=, % nothing happens
		stringstyle=\ttfamily, % typewriter type for strings
		showstringspaces=false} % no special string spaces

\frame{\titlepage}

\begin{frame}
Inaka Networks presents...
\pause
El Brujo Halc\'on in...
\pause
Scaling Erlang
\pause
(Based on a true story)
\end{frame}

\begin{frame}
A not so long time ago in a country far far away...
\end{frame}

\begin{frame}
Hey! Boca is playing at the Bombonera now!
Ok, let's watch it!
I can't, I'm at \emph{the office}
...
We need an app for that!
\end{frame}

\begin{frame}
... So we created MatchStream ...
\end{frame}

\begin{frame}
Ok, then... We know there will be hundreds of thousands of users, right?
We need the system to \alert{scale}
\pause
Of course! We should use Erlang!
\end{frame}

\begin{frame}
Several days after that...
\pause
MatchStream is ready!
TODO: INSERT SYSTEM DESCRIPTION HERE
\end{frame}

\begin{frame}
Boca plays again today, let's try it out with this game!
\pause
Wait, we can't handle more than 1000 users?! WTF?!?!
\pause
And they can only connect four at a time???
\end{frame}

\begin{frame}
Lesson Learned: Just using Erlang is not enough to make your system scale
\end{frame}

\begin{frame}
So... What did we do?
\end{frame}

\begin{frame}{Step 1}
We made sure the system was working.
\pause
\begin{itemize}
	\item<+-> We built a simulator
	\item<+-> We improved the logging mechanisms
	\item<+-> We tested the system
\end{itemize}
\end{frame}

\begin{frame}
1024 users / 4 at a time
\end{frame}

\begin{frame}{Step 2}
The system is fine, let's tune up the server where it's installed
\pause
So, we checked the kernel variables and system limits for
\begin{itemize}
	\item<+-> Concurrent TCP connections
	\item<+-> Open files limit
	\item<+-> TCP backlog size
	\item<+-> TCP memory allocation
	\item<+-> Erlang VM process limit
\end{itemize}
\end{frame}

\begin{frame}
4096 users / 4 at a time
\end{frame}

\begin{frame}{Step 3}
I've got a friend that may help us, he has a bag with several tips and tricks for us...
\pause
MacGyver
\end{frame}

\begin{frame}{Step 3}{Connection tweaks}
	\begin{description}
		\item<+->[Backlog]\ \\
			\begin{itemize}
				\item Allow more concurrent connections
				\item Remember HTTP \emph{runs on} TCP
			\end{itemize}
		\item<+->[Connections]\ \\
			\begin{itemize}
				\item Don't use just one of them
				\item Check inbound and outbound connections
			\end{itemize}
	\end{description}
\end{frame}

\begin{frame}
TODO users / TODO at a time
\end{frame}

\begin{frame}{Step 3}{gen\textunderscore event}
	\begin{description}
		\item<+->[sup\textunderscore handler]\ \\
			\begin{itemize}
				\item Don't use it
				\item Monitor the processes instead
			\end{itemize}
		\item<+->[Long Delivery Queues]\ \\
			\begin{itemize}
				\item Use \emph{repeaters}
			\end{itemize}
	\end{description}
\end{frame}

\begin{frame}
TODO users / TODO at a time
\end{frame}

\begin{frame}{Step 3}{gen\textunderscore server}
	\begin{description}
		\item<+->[Call Timeouts]\ \\
			Remember \texttt{gen\textunderscore server:reply/2}
		\item<+->[Memory Footprint]\ \\
			Remember \texttt{hibernate}
		\item<+->[Long \texttt{init/1}]\ \\
			Use $0$ timeout
	\end{description}
\end{frame}

\begin{frame}
TODO users / TODO at a time
\end{frame}

\lstset{language=erlang}
\defverbatim[colored]\startchild{%
\begin{lstlisting}[frame=single]
      supervisor:start_child(
      	list_to_atom("module-name_" ++
				integer_to_list(random:uniform(#ofSupervisors))).
\end{lstlisting}
}

\begin{frame}{Step 3}{supervisors}
	\begin{itemize}
		\item Sometimes \texttt{simple\textunderscore one\textunderscore for\textunderscore one} supervisors get \alert{overburdened} because they have too many children
		\item Try a supervisor hierarchy with several managers below the main supervisor
		\item Turn \texttt{supervisor:start\textunderscore child/2} calls into something like
		\startchild
	\end{itemize}
\end{frame}

\begin{frame}
TODO users / TODO at a time
\end{frame}

\begin{frame}{Step 3}{Other Processes}
	\begin{description}
		\item<+->[Timers]\ \\
			\begin{itemize}
				\item Don't use the \texttt{timer} module
				\item Use \texttt{erlang:send\textunderscore after}
			\end{itemize}
		\item<+->[Logging]\ \\
			\begin{itemize}
				\item Don't log too much
				\item Use a good logging system
			\end{itemize}
		\item<+->[Registration]\ \\
			\begin{itemize}
				\item Sometimes it's better to register processes instead of keeping track of their pids manually
				\item You can always register processes \alert{both} locally and globally
			\end{itemize}
	\end{description}
\end{frame}

\begin{frame}
64000 users / 8000 at a time
\end{frame}

\begin{frame}
TODO: Img of what the system looks like at this point
\end{frame}

\begin{frame}{Step 4}
Well, let's add some nodes to it!
\end{frame}

\begin{frame}{Step 4}{Adding Nodes}
Again, it's not as easy as just starting the app in another Erlang node
\pause
We needed to find the best topology, we considered using:
	\begin{itemize}
		\item connected nodes
		\item independent nodes
	\end{itemize}
\pause
We had to decide which processes needed to communicate and how
\pause
and of course, test the whole system again
\end{frame}

\begin{frame}
25000 users per node / 8000 per computer at a time
\pause
with 4 nodes on the same computer...
\pause
100K users / 8000 at a time
\end{frame}

\end{document}